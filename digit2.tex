%%%% Beállítások, importok

\input{includes/header.tex}

\lstset{language=[x86masm]Assembler} % kódrészlet színezéséhez
\graphicspath{{./images/}} % melyik mappában vannak a képek

%%%%
%%%%%%%
%%%% Ezeket változtasd meg!

\cim{Digitális technika 2. jegyzet}
\datum{2017. tavasz}
\szerzo{Sebők Bence}
\segitettek{} % Ha irsz bele valakit, majd torold ki a kommentet a 25. sorbol!

%%%%
%%%%%%%
%%%% Fedlap

\begin{document}
\begin{titlepage}
		\centering
		\vspace{5cm}\par
		\maketitle
		\large A jegyzet és annak forrása megtalálható a \texttt{bme-notes.github.io} weboldalon.
		\vfill
		% Közreműködtek: \the\segitettek
		\normalsize
		% Bottom of the page
\end{titlepage}

%%%%
%%%%%%%
%%%% Tartalomjegyzés + előszó

\noindent \textbf{Kellemes vizsgázást!}

\tableofcontents{}

\section{Előszó}
Ez a jegyzet a BME Digitális technika 2. tantárgyhoz szeretne segítséget nyújtani. Ezen kezdeményezés célja, hogy segítsen a hallgatóknak megérteni a tananyagot. A tantárgy nehéz, erősen ajánlott előadásra, gyakorlatra járni. Ez a jegyzet csak az előadáson, gyakorlaton készült jegyzet mellé jelent segítséget, nem tanít meg a nulláról a tantárgy minden apróságára.

Ez egy hallgatói jegyzet, nincs lektorálva, se egyetemi oktató által felügyelve. Amit itt olvasol, azt csak saját felelősségre használd, hivatalos helyeken nem hivatkozási alap ez a jegyzet.

%%%%
%%%%%%%
%%%% Tényleges content
%%%% A fejezeteket a fejezetek/$NEV.tex elérési útvonalon keresi
\ujfejezet{alapismeretek}
\ujfejezet{assembly}

\end{document}
